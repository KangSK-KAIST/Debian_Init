\documentclass[10pt]{article}

\usepackage{a4}
\usepackage{setspace}
\usepackage{ragged2e}
\usepackage[margin=1in]{geometry}
\usepackage{parskip}
\usepackage[english]{babel}
\usepackage{blindtext}
\usepackage{hyperref}
\usepackage{kotex}
\usepackage{amsmath, amssymb}
\usepackage{titling}
\usepackage{graphicx}
\usepackage{caption}
\usepackage{verbatim}
\usepackage{listings}
\usepackage{booktabs}
\usepackage{float}

\pretitle{\begin{center}\Huge}
\posttitle{\par\end{center}\vskip 0.5em}
\preauthor{\begin{flushright}\large\begin{tabular}[t]{c}}
\postauthor{\end{tabular}\par\vspace{-10pt}\end{flushright}}
\predate{\begin{flushright}\large}
\postdate{\par\end{flushright}}

\title{Manual for Shortcuts}
\author{Kang SK}
\date{October 28, 2020}

%% Start of Document
\begin{document}

\setlength{\parskip}{\baselineskip}
\setstretch{2.0}

\maketitle\label{title}
\tableofcontents\label{tableofcontents}
\newpage

\section*{Introduction} \label{S_Introduction}

\textbf{Shortcuts} are one of the best friends for a programmer, a writer, or basically anyone who works with a computer.
However, multiple environments have their own unique keybindings, making users hard to distinguish between them.
Of course, a user may fix his/her working environment, especially the working OS, to further fullfil the needs.
However, this may not be possible, or even if possible, a very tiresome job for some people, including myself.

Before going any further, I admit I used the Windows OS for quite a long time, and I still need to go back and forth from Linux to Windows.
Additionally, I admit, therefore I am more comfortable with the shortcuts on the Windows OS.
Although yes, I can get used to multiple sets of shortcuts, I don't really think this is necessary, if I can simply modify the settings instead.
This was the motivation for making this repository in the first place, and here it is, the manual for it.

Some might argue that this is a ``sacrilegious'' act, ignoring the design choices of bindings.
Some might argue that I am limiting myself from being able to use any other computer environment than my own.
I agree with some parts of the idea, but frankly saying, \emph{I don't care what others say}.

Therefore, I state here the objective of this manual.
First of all, this manual and all the shortcuts in it is based on \textbf{my} perspective, and is aiming for \textbf{my} convenience.
Modifying shortcuts of the Windows OS is much harder compared to changing settings on a Linux environment, therefore this manual is closer to the Windows' shortcuts.
This will result a Debian environment as convenient as possible for pre-Window users.

\newpage
\setstretch{1.5}

\section{Debian Shortcuts} \label{S_Debain}


\section{Terminal Shortcuts} \label{S_Terminal}


\section{VS Code Shortcuts} \label{S_VSCode}
	The basic philosophy of the shortcut design is as follows:

	\begin{itemize}
		\item Ctrl is the main function key.
		\item Up to two Ctrl keybindings make up functionality; the first key stands for the group, the second for the actual function.
		\item Window (Super) is for manipulating windows. Window could be thought of a alias for Ctrl + W.
		\item Alt is for alternation, which is used to alternate text.
	\end{itemize}

	\begin{figure}[H]
		\centering
		\begin{tabular}{@{}c c@{}}
			\toprule

			Toggle Fullscreen & Window + F \\
			Split Workspace & Window + \textbackslash \\
			Spilt Terminal & Window + \textbackslash \\

			\bottomrule
		\end{tabular}
		\caption{Keybindings for Windows}
	\end{figure}

	\begin{figure}[H]
		\centering
		\begin{tabular}{@{}c c@{}}
			\toprule

			Git Fetch & Ctrl + G, F \\
			Git Pull & Ctrl + G, Ctrl + P \\
			Git Commit & Ctrl + G, C \\
			Git Push & Ctrl + G, P \\
			Git Sync & Ctrl + G, S \\
			Git Merge & Ctrl + G, M \\
			Git Undo Commit & Ctrl + G, U \\
			Git Create Tag & Ctrl + G, T \\
			Git Delete Tag & Ctrl + G, Ctrl + T \\

			\bottomrule
		\end{tabular}
		\caption{Keybindings for Git}
	\end{figure}

\end{document}